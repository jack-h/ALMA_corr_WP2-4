\documentclass{article}
\usepackage[utf8]{inputenc}

\title{ALMA Correlator Study\\ WP2.4: Corner-turn platform}
\author{Jack Hickish}
\date{July 2016}

\begin{document}

\maketitle

\section{Introduction}
The ALMA Correlator Study assumes an upgraded ALMA correlator will be of an FX architecture, widely employed for digital correlators in radio astronomy.
Hence, the correlator will require a \emph{corner-turn}, or data transpose, between the F and X stages. Such a transpose allows data processing to be parallelised on a per-antenna basis in the F-stage and a per-frequency basis in the X stage.

In this document we consider the physical hardware used to implement the corner-turn, possible candidates being
\begin{itemize}
    \item Ethernet switches
    \item Custom back-plane
    \item Custom switching system
\end{itemize}

We begin by detailing the top-level specifications which determine the requirements of the corner-turning system. Specifications are taken from Rupen et al.

\section{Interconnect Specifications}
The specifications relevant to the corner-turning system discussed in this document are:

\begin{center}
\begin{tabular}{ c c c c }
 Rupen spec. & Description & Value & Symbol \\
 \hline
 2 & Number of antennas & up to 80 & $N$ \\
 5 & BBC Bandwidth & 14~GHz & $B$ \\
 4 & Number of BBCs per polarization & 1 & $n$ \\
 7 & Cross-correlation input bitwidth & 4+4 bit & $w$ \\
\end{tabular}
\end{center}

We make the following further assumptions:
\begin{enumerate}
\item The number of polarizations processed, $p$, is 2.
\item Correlators for each BBC are effectively independent. That is, the complete correlator may be constructed from $n$ distinct correlators.
\end{enumerate}

Considering a correlator for a single BBC, we may consider the inputs of the corner turner to be $pN$ F-Engines. Each F-Engine generates data at a rate $B \times w$ bits/s. For the assumed ALMA correlator, this is 112 Gb/s.

In the interests of simplicity and platform agnosticism we consider the $pN$ corner-turner inputs to be physically separate. However, each of these 112~Gb/s streams may be split over parallel interfaces in order to be practically feasible. For example, a 112Gb/s stream may be implemented using 3 parallel 40~Gb/s Ethernet interfaces, or via 12 10~Gb/s serial links.

Note that assuming that all $pN$ correlator inputs are physically separate increases the complexity of the corner turner, but minimally constraints the F-Engines. In practice it may be possible to combine multiple antennas into a single F-processor, reducing the number of inputs to the corner-turner (with a corresponding increase in bandwidth of each input). 

The output data-rate of the corner-turner is the same as the input, i.e., $pN \times 112$~Gb/s. 

\section{F-Engine interface}
In this document we assume an F-Engine processes data from a single BBC for a single polarization of a single antenna. The total ALMA correlator has $N_f = pN$ F-Engines. We assume nothing about the F-Engine interface except that it is capable of outputting a total of 112~Gb/s over $f$ independent links. Where $f>1$, we assume that the mulitple outputs contain sub-bands of the total processed bandwidth. In essence, the downstream corner-turner and correlator is then $f$ clones of a smaller correlator processing a bandwidth $\frac{B}{f}$.

\section{X-Engine interface}
We assume that a complete correlator system comprises $N_x$ X-Engines, each processing a subset of the total correlator bandwidth. The number of X-Engines is determined by the computational performance of a single unit, and need not be related to the number of F-Engines, $pN$. There is no assumed requirement on the X-engine interface, other than a single X-Engine is capable of sinking its fraction of the total system bandwidth: $\frac{pNBw}{N_x}$. This may be achieved via a single wide-band link, or via $x$ multiple parallel links.

Where the connection between the $N_f$ F-Engines and $N_x$ X-Engines is not direct (because, for example, it is mediated by an Ethernet switch) there is no requirement that the protocol of the F- and X-Engine interfaces should be the same.

\section{Ethernet Switch}
An Ethernet switch has a variety of attractive attributes:
\begin{itemize}
    \item No hardware NRE.
    \item Industry-standard interface, widely supported by commodity hardware (FPGA boards, CPU/GPU platforms).
    \item Extremely tolerant to changes in F- and X-Engine implementations or changes to number of antennas.
    \item Trivially supports hardware testing via CPU-driven test-vector injection.
    \item May supports conversion between protocols -- 10/40/100~GbE.
\end{itemize}

Though the available Ethernet technology at the time of deployment of a new correlator is uncertain, we can demonstrate the feasibility of an Ethernet corner-turn solution based on 2016 technology and assume that future solutions will be cheaper and denser.

We hypothesize the following system:
\begin{itemize}
    \item $N_f = pN = 160$ F-Engines
    \item F-Engine output protocol is 100~GbE, with $f=2$ independent interfaces, each carrying $\frac{112}{2} = 56$~Gb/s.
    \item $N_x = 256$ X-Engines, each processing 89.6~Gb/s, over a single 100~GbE link.
\end{itemize}

Such a system requires, for each of the two BBCs, two duplicates of a corner-turner with 112 F-Engine inputs, and 128 X-Engine outputs. Thus, the corner-turner is a switch with at least $128+112=240$ 100~GbE ports. Such switches are available off-the-shelf today. For example, the Arista 7508R with up to 288 100~GbE ports.

Alternatively, one can construct a larger switch from smaller modules, which can have more predictable performance for the all-to-all corner-turn systems required by correlators. In this case, the same system can be achieved with 8 individual 64-port 100~GbE switches, such as the 7260CX-64.

For other interfacing standards, such as 40~GbE using F-Engines with $f=3$ or $f=4$ output ports, similar systems can be constructed using 40~GbE switches.

\end{document}

